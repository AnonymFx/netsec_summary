%!TEX root = ../report.tex

\section{Language-theoretic Security}
Communication protocols define the procedure and the format of exchanged messages.
If two communication partners speak the same protocol it is not necessarily given that they have the same understanding though.
Different problems may arise which result in some guidelines to avoid them:

\begin{description}
  \item[Full input recognition] Programmers usually assume well formed input where in reality the input is controlled by the attacker.
    For this reason every input should be checked if it fulfills the expectations of valid inputs.
  \item[Only Type 2 or 3 of Chomsky Hierarchy grammars for inputs]
    Do not define turing-complete protocols because recognition of the input and testing the equivalence of implementations is undecidable.
    \begin{figure}[h]
      \centering
      \begin{tabular}{l l l}
        Grammar & Language & Recognized by\\
        \midrule
        Type 3 & Regular & Finite state automaton\\
        Type 2 & Context-free & Pushdown automaton\\
        Type 1 & Context-sensitive & Some weird stuff\\
        Type 0 & recursively enumerable & Turing machine\\
      \end{tabular}
      \caption{Chomsky Hierarchy of Languages}
    \end{figure}
  \item[Reduce computing power] Reduce the computing power exposed to the outside.
    Computing power that is not there can not be exploited.
    This also includes defining protocols only in type 2 or 3 languages since parsing types 0 and 1 requires large or potentially unlimited amounts of computing.
  \item[Same interpretation of messages] All participants of a communication should interpret messages the same way.
    For this parsers have to be equivalent which is only decidable for type 2 and 3.
\end{description}

